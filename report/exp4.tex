%!TeX program = xelatex
\documentclass[12pt,hyperref,a4paper,UTF8]{ctexart}
\usepackage{zjureport}
\usepackage{listings}
\usepackage{xcolor}

\lstdefinelanguage{Rust}{
  morekeywords={fn, let, mut, use, impl, pub, struct, enum, match, for, in, if, else, loop, while, break, continue, return, Some, None, Option},
  sensitive=true,
  morecomment=[l]{//},
  morestring=[b]"
}
\lstset{
  language=Rust,
  basicstyle=\ttfamily\small,
  keywordstyle=\color{blue},
  commentstyle=\color{gray},
  stringstyle=\color{red},
  breaklines=true,
  numbers=left,
  numberstyle=\tiny,\color{gray},
  frame=single,
  tabsize=2,
}

\begin{document}

% 自定义封面(实验四)
\begin{titlepage}
  \centering
  {\zihao{2}\textbf{实验四:内存页面淘汰算法(命中率)实验}}\\[2em]
  {\zihao{4}课程:操作系统\\}
  \vfill
    \begin{center}
      \zihao{-4}姓名:赖景康\\ 
      学号: 2023214564\\ 
      指导教师: 赵玉琦\\ 
      提交日期: 2025.12.03\\ 
    \end{center}
\end{titlepage}

\newpage

\section{实验目的}
\begin{itemize}
  \item 理解并实现三种主流页面置换算法:FIFO、LRU 与 OPT(最佳算法)并比较其缺页性能;
  \item 探索页框数量变化对缺页率的影响(尤其观察 Belady 现象);
  \item 能用程序模拟不同算法并统计缺页次数,为理解页面置换策略提供直观实验数据。
\end{itemize}

\section{问题描述与实验要求}
本实验针对同一批参考串,比较 FIFO、LRU、OPT 三种页面置换算法在不同页框数下的缺页次数;
并对比 FIFO/LRU 在 k=2,3,4 下的缺页情况;以及构造 Belady 现象的示例序列,验证 FIFO 在增加页框时可能出现缺页次数反而增加的异常情况。

\section{实验环境}
\begin{itemize}
  \item 开发语言:Rust
  \item 开发工具:VS Code
  \item 编译/运行:Cargo(注:在当前编辑环境中无法直接运行 Cargo,我在报告中给出解析结果;建议在本地运行 `cargo run` 来获取控制台输出并替换本文中的数值以作核验。)
\end{itemize}

\section{实现思路与主要函数}
实现采用单线程事件驱动式模拟:给定页面访问参考串与页框数,分别实现三种算法的模拟函数,返回缺页次数。主要函数签名如下(节选自 `task4/src/main.rs`):
\begin{verbatim}
fn simulate_fifo(reference: &[i32], frames: usize) -> usize
fn simulate_lru(reference: &[i32], frames: usize) -> usize
fn simulate_opt(reference: &[i32], frames: usize) -> usize
\end{verbatim}

实现要点:
\begin{itemize}
  \item FIFO 使用一个队列(VecDeque)维护页面入框顺序,缺页时弹出队头并加入新页;
  \item LRU 通过维护每个页的最后使用时间(索引)来选取最近最久未使用页进行替换;
  \item OPT 在发生缺页且页框已满时,对在当前点之后每个在内页的下一次出现位置做查找,选择下一次出现最远的页进行替换;若某页不再出现则优先替换它(实现为寻找 position 返回 None 的帧)。
\end{itemize}

\section{代码展示(节选)}
下面给出程序中用于模拟三种页面置换算法的关键实现节选(完整代码见 `task4/src/main.rs`)。
\begin{lstlisting}
use std::collections::{HashMap, HashSet, VecDeque};

fn simulate_fifo(reference: &[i32], frames: usize) -> usize {
  assert!(frames > 0, "页框数必须大于 0");
  let mut faults = 0;
  let mut frame_queue: VecDeque<i32> = VecDeque::new();
  let mut in_memory: HashSet<i32> = HashSet::new();

  for &page in reference {
    if in_memory.contains(&page) { continue; }
    faults += 1;
    if frame_queue.len() == frames {
      if let Some(evicted) = frame_queue.pop_front() {
        in_memory.remove(&evicted);
      }
    }
    frame_queue.push_back(page);
    in_memory.insert(page);
  }
  faults
}
\end{lstlisting}

\begin{lstlisting}
fn simulate_lru(reference: &[i32], frames: usize) -> usize {
  assert!(frames > 0, "页框数必须大于 0");
  let mut faults = 0;
  let mut frames_vec: Vec<i32> = Vec::new();
  let mut last_used: HashMap<i32, usize> = HashMap::new();

  for (idx, &page) in reference.iter().enumerate() {
    if frames_vec.contains(&page) {
      last_used.insert(page, idx);
      continue;
    }
    faults += 1;
    if frames_vec.len() < frames {
      frames_vec.push(page);
    } else {
      let (replace_idx, _) = frames_vec
        .iter()
        .enumerate()
        .min_by_key(|(_, &p)| last_used.get(&p).copied().unwrap_or(0))
        .expect("至少有一个页框");
      frames_vec[replace_idx] = page;
    }
    last_used.insert(page, idx);
  }
  faults
}
\end{lstlisting}

\begin{lstlisting}
fn simulate_opt(reference: &[i32], frames: usize) -> usize {
  assert!(frames > 0, "页框数必须大于 0");
  let mut faults = 0;
  let mut frames_vec: Vec<i32> = Vec::new();

  for (idx, &page) in reference.iter().enumerate() {
    if frames_vec.contains(&page) { continue; }
    faults += 1;
    if frames_vec.len() < frames {
      frames_vec.push(page);
      continue;
    }
    let mut target_idx = 0;
    let mut farthest_distance: Option<usize> = None;
    for (frame_idx, &frame_page) in frames_vec.iter().enumerate() {
      match reference[idx + 1..].iter().position(|&p| p == frame_page) {
        None => { target_idx = frame_idx; break; }
        Some(dist) => {
          let dist = dist + 1;
          if farthest_distance.map_or(true, |current| dist > current) {
            farthest_distance = Some(dist);
            target_idx = frame_idx;
          }
        }
      }
    }
    frames_vec[target_idx] = page;
  }
  faults
}
\end{lstlisting}


\section{实验数据与参考串}
本次实验使用程序中给定的参考串(见 `task4/src/main.rs`):
\begin{verbatim}
reference = [1,2,3,2,4,1,5,2,1,2,3,4,5,2,1]
\end{verbatim}
以及用于演示 Belady 现象的构造序列:
\begin{verbatim}
belady_sequence = [1,2,3,4,1,2,5,1,2,3,4,5]
\end{verbatim}

\section{实验结果(解析计算)}
注:由于当前环境无法直接执行 `cargo run`,下面的缺页次数为对 `task4/src/main.rs` 中算法逻辑的逐步模拟/分析计算得到的结果;你可以在本地运行程序以验证并把控制台输出粘贴回本报告。

\subsection{同一批数据下(页框 = 3)}
\begin{itemize}
  \item FIFO 缺页次数:12
  \item LRU 缺页次数:12
  \item OPT 缺页次数(参考):8
\end{itemize}

\subsection{FIFO 在不同页框数下(k=2,3,4)}
\begin{table}[h]
\centering
\begin{tabular}{c|c|c|c}
页框数 & 2 & 3 & 4 \\
\hline
缺页次数 & 13 & 12 & 11 \\
\end{tabular}
\caption{FIFO 在不同页框数下的缺页次数}
\end{table}

\subsection{LRU 在不同页框数下(k=2,3,4)}
\begin{table}[h]
\centering
\begin{tabular}{c|c|c|c}
页框数 & 2 & 3 & 4 \\
\hline
缺页次数 & 13 & 12 & 9 \\
\end{tabular}
\caption{LRU 在不同页框数下的缺页次数}
\end{table}

\subsection{Belady 现象示例(FIFO)}
对构造序列 `belady_sequence`,FIFO 在不同页框下的缺页次数为:
\begin{itemize}
  \item 页框=3:缺页次数 = 9
  \item 页框=4:缺页次数 = 10
\end{itemize}
这说明在该构造序列下,FIFO 随着页框数增加反而导致缺页次数上升,即出现 Belady 异常。

\section{图表与表格展示}
为了避免对额外 LaTeX 包的依赖(如 `pgfplots`),下面以表格形式展示统计数据,便于在任何标准 TeX 安装下编译。

\begin{table}[h]
\centering
\begin{tabular}{l|r}
算法 & 缺页次数 (页框=3) \\
\hline
FIFO & 12 \\
LRU  & 12 \\
OPT  & 8 \\
\end{tabular}
\caption{同一批数据下 (页框=3) 三种算法缺页次数比较}
\end{table}

\begin{table}[h]
\centering
\begin{tabular}{c|c|c}
页框数 & FIFO & LRU \\
\hline
2 & 13 & 13 \\
3 & 12 & 12 \\
4 & 11 & 9  \\
\end{tabular}
\caption{FIFO 与 LRU 在不同页框数下的缺页次数比较}
\end{table}

\begin{table}[h]
\centering
\begin{tabular}{c|r}
页框数 & FIFO (Belady 序列) \\
\hline
3 & 9 \\
4 & 10 \\
\end{tabular}
\caption{Belady 构造序列下 FIFO 缺页次数(页框 3 vs 4)}
\end{table}

\section{分析与讨论}
\begin{itemize}
  \item OPT(最佳算法)由于能预知未来,缺页次数最低;在本例中,OPT(3) = 8,显著优于 FIFO/LRU(3)=12。
  \item 对于该参考串,FIFO 与 LRU 在 k=3 时得到相同的缺页次数,但在 k=4 时 LRU 明显优于 FIFO(LRU(4)=9,FIFO(4)=11),体现出 LRU 更善于利用局部性。
  \item Belady 现象:FIFO 在构造序列下出现缺页次数随页框增加反而增加的情况(3 -> 4:9 -> 10),说明 FIFO 不是单调最优的;而 LRU/OPT 不会出现此类异常(LRU 通常避免 Belady)。
  \item 实验提示:在实际系统中应优先使用带有最近使用信息的替换策略(如 LRU 或其近似算法)以获得更稳定的命中率;简单的 FIFO 尽管实现简单,但可能在某些工作集上表现极差。
\end{itemize}

\section{实验结论}
本次实验通过对比 FIFO、LRU、OPT 三种页面置换算法,验证了:
\begin{itemize}
  \item OPT 为理论最优;
  \item LRU 在多数具有局部性的参考串上优于 FIFO;
  \item FIFO 可能会发生 Belady 现象,即增加页框反而增加缺页次数;因此实际系统中应谨慎采用单纯的 FIFO 替换策略。
\end{itemize}

\section{附录:关键源代码(节选)}
以下为 `task4/src/main.rs` 的主要实现节选:
\begin{verbatim}
use std::collections::{HashMap, HashSet, VecDeque};

fn simulate_fifo(reference: &[i32], frames: usize) -> usize { ... }

fn simulate_lru(reference: &[i32], frames: usize) -> usize { ... }

fn simulate_opt(reference: &[i32], frames: usize) -> usize { ... }

fn main() {
  let reference = vec![1,2,3,2,4,1,5,2,1,2,3,4,5,2,1];
  // ...
}
\end{verbatim}

如需完整代码,请参见仓库路径:`task4/src/main.rs`(已附在项目中)。

\section{参考文献}
\begin{itemize}
  \item Belady, L. A. (1966). A study of replacement algorithms for a virtual-storage computer. (经典页面置换问题论文)
  \item 教材与讲义:操作系统课程相关讲义(参见课程资料)
  \item Github 仓库:\url{https://github.com/shufufufu/OS-TASKS}
\end{itemize}

\end{document}
