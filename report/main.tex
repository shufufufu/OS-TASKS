
%!TeX program = xelatex
\documentclass[12pt,hyperref,a4paper,UTF8]{ctexart}
\usepackage{zjureport}

%%-------------------------------正文开始---------------------------%%
\begin{document}

%%-----------------------封面--------------------%%
\cover

%%------------------摘要-------------%%
%\begin{abstract}
%
%在此填写摘要内容
%
%\end{abstract}

\thispagestyle{empty} % 首页不显示页码

%%--------------------------目录页------------------------%%
% \newpage
% \tableofcontents

%%------------------------正文页从这里开始-------------------%
\newpage

%%可选择这里也放一个标题
%\begin{center}
%    \title{ \Huge \textbf{{标题}}}
%\end{center}

\section{实验目的和要求}
\begin{itemize}
    \item 掌握周转时间、等待时间、平均周转时间等概念及其计算方法。
    \item 理解三种常用的进程调度算法,区分算法之间的差异性,并模拟实现各算法。
    \item 了解操作系统中高级调度、中级调度和低级调度的区别和联系
\end{itemize}

\section{问题描述}
    \begin{itemize}
        \item (1)在单道环境下,已知 $n$个作业的进入时间和估计运行时间(以分钟计),分别求出每一个作业的开始时间、结束时间、周转时间、带权周转时间,以及这些作业的平均周转时间和带权平均周转时间;
        \item (2)在多道环境(如 2 道)下,已知 $n$ 个作业的进入时间和估计运行时间(以分钟计),分别求出每一个作业的开始时间、结束时间、周转时间、带权周转时间,以及这些作业的平均周转时间和带权平均周转时间。
    \end{itemize}

\section{实验要求}
    \begin{itemize}
        \item 分别用先来先服务调度算法(FCFS)、短作业优先调度算法(SJF)、响应比高者优先调度算法(HRRN),求出批作业的平均周转时间和带权平均周转时间;
        \item 就同一批次作业,分别讨论这些算法的优劣;
        \item 衡量同一调度算法对不同作业流的性能。
    \end{itemize}


\section{实验环境}
    \begin{itemize}
        \item 开发工具:VS Code
        \item 编程语言: Rust
    \end{itemize}

\section{设计思想及实验步骤}
(包括实验设计原理,分析方法、计算步骤、模块组织,或主要流程图、伪代码等)

\subsection{实验设计原理}
本实验采用非抢占式批处理作业调度模型,核心指标:周转时间 \(T_i=C_i-A_i\)、带权周转时间 \(W_i=T_i/S_i\),以及其平均值 \(\overline{T}=\frac{1}{n}\sum_i T_i\)、\(\overline{W}=\frac{1}{n}\sum_i W_i\)。多道情形抽象为 \(m\) 条并行“道”,作业一旦开始在某道上运行至完成。

\subsection{分析方法}
实现三种典型非抢占调度算法:FCFS(按到达时间先后分配到最早空闲道)、SJF(就绪集合选服务时间最短者)、HRRN(就绪集合按 \(R=(t-A+S)/S\) 从大到小选择)。时间推进遵循事件驱动:若有空闲道且就绪非空则立即分配,否则跳至下一个到达或最近空闲时间的较小者。

\subsection{计算步骤}
\begin{enumerate}
    \item 构造作业流 \((A_i,S_i)\),设定道数 \(m\);
    \item 维护就绪集合与各道最早空闲时间;
    \item 依据所选算法分配作业并记录开始/结束时间;
    \item 计算 \(T_i, W_i\) 及 \(\overline{T}, \overline{W}\);
    \item 输出并整理为表格,进行对比分析。
\end{enumerate}

\subsection{模块组织}
Rust程序采用面向对象设计,主要包含以下模块:
\begin{itemize}
    \item \textbf{Task结构体}:定义任务的基本属性(任务ID、到达时间、执行时间、开始时间、完成时间)及相关方法;
    \item \textbf{调度算法函数}:\texttt{execute\_fcfs\_scheduling}、\texttt{execute\_sjf\_scheduling}、\texttt{execute\_hrrn\_scheduling}分别实现三种调度算法;
    \item \textbf{结果处理函数}:\texttt{display\_scheduling\_results}负责统计与格式化输出调度结果;
    \item \textbf{测试数据生成}:\texttt{generate\_test\_task\_set\_1}、\texttt{generate\_test\_task\_set\_2}生成不同特征的任务流;
    \item \textbf{主程序}:\texttt{main}函数组装单处理器/双处理器环境与两组任务流的完整实验流程。
\end{itemize}

\subsection{伪代码}
以SJF算法为例(非抢占式,多处理器环境):
\begin{verbatim}
current_time <- 0; processor_available_time[m] <- 0; ready_queue <- {}
按到达时间排序 all_tasks
while 未完成任务:
  将 arrival_time <= current_time 的任务加入 ready_queue
  available_processors <- {k | processor_available_time[k] <= current_time}
  if available_processors 为空:
    if ready_queue 非空: current_time <- 最早 processor_available_time
    else: current_time <- min(下一个到达, 最早 processor_available_time); continue
  if ready_queue 为空: current_time <- 下一个到达; continue
  按 execution_time 升序排序 ready_queue
  对每个空闲处理器:
    取最短任务 task;start_time <- current_time; finish_time <- start_time + execution_time
    记录 task.start_time/task.finish_time;processor_available_time[processor] <- finish_time;加入 completed_tasks
  current_time <- min(下一个到达, 最早 processor_available_time)
\end{verbatim}

\section{实验结果及分析}
以下给出程序运行结果(单位:分钟)。

\paragraph{程序运行结果}
\begin{verbatim}
操作系统作业调度算法实验
================================

【单处理器环境调度结果】

========== FCFS算法 - 单处理器 ==========
任务ID\t到达\t执行\t开始\t结束\t周转\t带权周转
1\t0.00\t4.00\t0.00\t4.00\t4.00\t1.00
2\t1.50\t7.00\t4.00\t11.00\t9.50\t1.36
3\t3.00\t5.00\t11.00\t16.00\t13.00\t2.60
4\t5.00\t6.00\t16.00\t22.00\t17.00\t2.83
5\t7.00\t3.00\t22.00\t25.00\t18.00\t6.00
平均周转时间: 12.3000
平均带权周转时间: 2.7580

========== SJF算法 - 单处理器 ==========
任务ID\t到达\t执行\t开始\t结束\t周转\t带权周转
1\t0.00\t4.00\t0.00\t4.00\t4.00\t1.00
2\t1.50\t7.00\t4.00\t11.00\t9.50\t1.36
3\t3.00\t5.00\t13.00\t18.00\t15.00\t3.00
4\t5.00\t6.00\t18.00\t24.00\t19.00\t3.17
5\t7.00\t3.00\t11.00\t14.00\t7.00\t2.33
平均周转时间: 10.9000
平均带权周转时间: 2.1720

========== HRRN算法 - 单处理器 ==========
任务ID\t到达\t执行\t开始\t结束\t周转\t带权周转
1\t0.00\t4.00\t0.00\t4.00\t4.00\t1.00
2\t1.50\t7.00\t4.00\t11.00\t9.50\t1.36
3\t3.00\t5.00\t11.00\t16.00\t13.00\t2.60
4\t5.00\t6.00\t18.00\t24.00\t19.00\t3.17
5\t7.00\t3.00\t16.00\t19.00\t12.00\t4.00
平均周转时间: 11.5000
平均带权周转时间: 2.4260

【双处理器环境调度结果】

========== FCFS算法 - 双处理器 ==========
任务ID\t到达\t执行\t开始\t结束\t周转\t带权周转
1\t0.00\t4.00\t0.00\t4.00\t4.00\t1.00
2\t1.50\t7.00\t1.50\t8.50\t7.00\t1.00
3\t3.00\t5.00\t3.00\t8.00\t5.00\t1.00
4\t5.00\t6.00\t8.00\t14.00\t9.00\t1.50
5\t7.00\t3.00\t8.50\t11.50\t4.50\t1.50
平均周转时间: 6.0000
平均带权周转时间: 1.2000

========== SJF算法 - 双处理器 ==========
任务ID\t到达\t执行\t开始\t结束\t周转\t带权周转
1\t0.00\t4.00\t0.00\t4.00\t4.00\t1.00
2\t1.50\t7.00\t1.50\t8.50\t7.00\t1.00
3\t3.00\t5.00\t3.00\t8.00\t5.00\t1.00
4\t5.00\t6.00\t8.00\t14.00\t9.00\t1.50
5\t7.00\t3.00\t8.50\t11.50\t4.50\t1.50
平均周转时间: 6.0000
平均带权周转时间: 1.2000

========== HRRN算法 - 双处理器 ==========
任务ID\t到达\t执行\t开始\t结束\t周转\t带权周转
1\t0.00\t4.00\t0.00\t4.00\t4.00\t1.00
2\t1.50\t7.00\t1.50\t8.50\t7.00\t1.00
3\t3.00\t5.00\t3.00\t8.00\t5.00\t1.00
4\t5.00\t6.00\t8.00\t14.00\t9.00\t1.50
5\t7.00\t3.00\t8.50\t11.50\t4.50\t1.50
平均周转时间: 6.0000
平均带权周转时间: 1.2000

【算法性能对比分析】
================================

========== 任务流A - FCFS算法 ==========
任务ID\t到达\t执行\t开始\t结束\t周转\t带权周转
1\t0.00\t4.00\t0.00\t4.00\t4.00\t1.00
2\t1.50\t7.00\t4.00\t11.00\t9.50\t1.36
3\t3.00\t5.00\t11.00\t16.00\t13.00\t2.60
4\t5.00\t6.00\t16.00\t22.00\t17.00\t2.83
5\t7.00\t3.00\t22.00\t25.00\t18.00\t6.00
平均周转时间: 12.3000
平均带权周转时间: 2.7580

========== 任务流B - FCFS算法 ==========
任务ID\t到达\t执行\t开始\t结束\t周转\t带权周转
1\t0.00\t9.00\t0.00\t9.00\t9.00\t1.00
2\t0.50\t3.00\t9.00\t12.00\t11.50\t3.83
3\t1.00\t10.00\t12.00\t22.00\t21.00\t2.10
4\t2.00\t4.00\t22.00\t26.00\t24.00\t6.00
5\t8.00\t1.00\t26.00\t27.00\t19.00\t19.00
6\t8.50\t2.00\t27.00\t29.00\t20.50\t10.25
平均周转时间: 17.5000
平均带权周转时间: 7.0300

实验完成!
\end{verbatim}

\subsection{单道(m=1)}
\paragraph{FCFS}
\begin{table}[!htbp]
\centering
\begin{tabular}{c|c|c|c|c|c|c}
任务ID & 到达 & 执行 & 开始 & 结束 & 周转 & 带权周转 \\
\hline
1 & 0.00 & 4.00 & 0.00 & 4.00 & 4.00 & 1.00 \\
2 & 1.50 & 7.00 & 4.00 & 11.00 & 9.50 & 1.36 \\
3 & 3.00 & 5.00 & 11.00 & 16.00 & 13.00 & 2.60 \\
4 & 5.00 & 6.00 & 16.00 & 22.00 & 17.00 & 2.83 \\
5 & 7.00 & 3.00 & 22.00 & 25.00 & 18.00 & 6.00 \\
\hline
\multicolumn{7}{r}{\small 平均周转时间 = 12.3000,平均带权周转时间 = 2.7580}
\end{tabular}
\end{table}

\paragraph{SJF}
\begin{table}[!htbp]
\centering
\begin{tabular}{c|c|c|c|c|c|c}
任务ID & 到达 & 执行 & 开始 & 结束 & 周转 & 带权周转 \\
\hline
1 & 0.00 & 4.00 & 0.00 & 4.00 & 4.00 & 1.00 \\
2 & 1.50 & 7.00 & 4.00 & 11.00 & 9.50 & 1.36 \\
3 & 3.00 & 5.00 & 13.00 & 18.00 & 15.00 & 3.00 \\
4 & 5.00 & 6.00 & 18.00 & 24.00 & 19.00 & 3.17 \\
5 & 7.00 & 3.00 & 11.00 & 14.00 & 7.00 & 2.33 \\
\hline
\multicolumn{7}{r}{\small 平均周转时间 = 10.9000,平均带权周转时间 = 2.1720}
\end{tabular}
\end{table}

\paragraph{HRRN}
\begin{table}[!htbp]
\centering
\begin{tabular}{c|c|c|c|c|c|c}
任务ID & 到达 & 执行 & 开始 & 结束 & 周转 & 带权周转 \\
\hline
1 & 0.00 & 4.00 & 0.00 & 4.00 & 4.00 & 1.00 \\
2 & 1.50 & 7.00 & 4.00 & 11.00 & 9.50 & 1.36 \\
3 & 3.00 & 5.00 & 11.00 & 16.00 & 13.00 & 2.60 \\
4 & 5.00 & 6.00 & 18.00 & 24.00 & 19.00 & 3.17 \\
5 & 7.00 & 3.00 & 16.00 & 19.00 & 12.00 & 4.00 \\
\hline
\multicolumn{7}{r}{\small 平均周转时间 = 11.5000,平均带权周转时间 = 2.4260}
\end{tabular}
\end{table}

\subsection{双处理器环境(m=2)}
三种算法在该任务流下得到相同的调度与指标:
\begin{table}[!htbp]
\centering
\begin{tabular}{c|c|c|c|c|c|c}
任务ID & 到达 & 执行 & 开始 & 结束 & 周转 & 带权周转 \\
\hline
1 & 0.00 & 4.00 & 0.00 & 4.00 & 4.00 & 1.00 \\
2 & 1.50 & 7.00 & 1.50 & 8.50 & 7.00 & 1.00 \\
3 & 3.00 & 5.00 & 3.00 & 8.00 & 5.00 & 1.00 \\
4 & 5.00 & 6.00 & 8.00 & 14.00 & 9.00 & 1.50 \\
5 & 7.00 & 3.00 & 8.50 & 11.50 & 4.50 & 1.50 \\
\hline
\multicolumn{7}{r}{\small 平均周转时间 = 6.0000,平均带权周转时间 = 1.2000}
\end{tabular}
\end{table}

\subsection{讨论与分析}
\begin{itemize}
    \item 单处理器环境下:SJF算法的平均周转时间和平均带权周转时间最低(10.9000和2.1720),体现了短作业优先的优势;HRRN算法介于FCFS与SJF之间(11.5000和2.4260),能够有效缓解长作业饥饿问题;
    \item 双处理器环境下:由于任务到达时间和执行时间的结构特点,本例中三种算法输出结果一致,且显著优于单处理器环境(平均周转时间从12.3000降至6.0000);
    \item 多处理器并行能够显著降低任务等待时间;在实际操作系统中,若考虑I/O操作、抢占式调度、优先级等因素,调度策略需要进一步扩展和优化。
\end{itemize}

\section{附录:部分源代码}
\begin{verbatim}
use std::cmp::Ordering;

/// 作业任务结构体
/// 表示一个需要调度的作业任务
#[derive(Clone, Debug)]
struct Task {
    task_id: usize,        // 任务标识符
    arrival_time: f64,     // 任务到达时间(分钟)
    execution_time: f64,   // 预估执行时间(分钟)
    start_time: Option<f64>,  // 实际开始时间
    finish_time: Option<f64>, // 实际完成时间
}

impl Task {
    /// 创建新的任务实例
    fn create(task_id: usize, arrival_time: f64, execution_time: f64) -> Self {
        Self { 
            task_id, 
            arrival_time, 
            execution_time, 
            start_time: None, 
            finish_time: None 
        }
    }

    /// 计算任务周转时间
    fn calculate_turnaround_time(&self) -> Option<f64> {
        match (self.finish_time, Some(self.arrival_time)) {
            (Some(finish), Some(arrival)) => Some(finish - arrival),
            _ => None,
        }
    }

    /// 计算任务带权周转时间
    fn calculate_weighted_turnaround_time(&self) -> Option<f64> {
        match (self.calculate_turnaround_time(), self.execution_time) {
            (Some(turnaround), exec_time) if exec_time > 0.0 => Some(turnaround / exec_time),
            _ => None,
        }
    }
}

/// 先来先服务调度算法实现
/// 按照任务到达时间顺序进行调度分配
fn execute_fcfs_scheduling(tasks: &[Task], processor_count: usize) -> Vec<Task> {
    let mut task_list: Vec<Task> = tasks.to_vec();
    // 按到达时间排序
    task_list.sort_by(|a, b| a.arrival_time.partial_cmp(&b.arrival_time).unwrap_or(Ordering::Equal));

    // 记录每个处理器的空闲时间
    let mut processor_available_time: Vec<f64> = vec![0.0; processor_count];

    for task in &mut task_list {
        // 找到最早空闲的处理器
        let (processor_index, &available_time) = processor_available_time.iter().enumerate()
            .min_by(|a, b| a.1.partial_cmp(b.1).unwrap_or(Ordering::Equal)).unwrap();
        
        // 任务开始时间 = max(到达时间, 处理器空闲时间)
        let start_time = task.arrival_time.max(available_time);
        let finish_time = start_time + task.execution_time;
        
        task.start_time = Some(start_time);
        task.finish_time = Some(finish_time);
        processor_available_time[processor_index] = finish_time;
    }
    task_list
}
\end{verbatim}


\section{写在最后}
\subsection{发布地址}
\begin{itemize}
    \item Github: \url{https://github.com/shufufufu/OS-TASKS}
\end{itemize}

%%----------- 参考文献 -------------------%%
%在reference.bib文件中填写参考文献,此处自动生成

% \reference


\end{document}